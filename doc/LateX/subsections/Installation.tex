\subsubsection{MySQL}
Für das Nutzersystem wird das Datenbanksystem MySQL benötigt. Um MySQL auf Linux zu installieren sollten die folgenden Schritte befolgt werden:
\begin{enumerate}
	\item sudo apt-update
	\item sudo apt install mysql-server
	\item sudo systemctl start mysql.service
\end{enumerate}
Dann muss die Passwortauthentifizierung für MySQL aktiviert werden da der Webserver sonst keine Daten aus der Datenbank lesen oder Daten in die Datenbank schreiben kann.
\begin{enumerate}
	\item sudo mysql \newpage
	\item ALTER USER 'root'@'localhost' IDENTIFIED WITH mysql\_native\_password BY 'password';
	\item exit
	\item sudo mysql\_secure\_installation
	\item mysql -u root -p
	\item ALTER USER 'root'@'localhost' IDENTIFIED WITH auth\_socket;
\end{enumerate}
Aktuell läuft das System über den root Benutzer des Datenbanksystems, was nicht optimal ist, da es einfach wäre die Datenbank anzugreifen.
\cite{MySQLinstall}

\subsubsection{Nodejs}
Die Website an sich ist geschrieben mit und wird gehostet von NodeJS. Das heißt auf dem Server muss auch NodeJS installiert werden. \cite{Nodejsinstall}
\begin{enumerate}
	\item sudo apt update
	\item sudo apt install nodejs
	\item node -v
	\item sudo apt install npm
\end{enumerate}


\subsubsection{Website Relevante Daten}
Um die Website später Hosten zu können müssen die Dateien im src/ Ordner des Repositories auf dem Linux Server liegen. Der einfachste Weg sollte ein standard Git Clone Befehl sein.
\begin{enumerate}
	\item git clone ‚link-to-repository-here‘
\end{enumerate}
Git sollte bereits teil der Ubuntu Distribution sein. \newline
Daraufhin sollte in den src/ Ordner des Repositories navigiert werden.
\begin{enumerate}[resume]
	\item Cd repositoryName/src
	\item npm install
\end{enumerate}
Jetzt muss die Datenbank initialisiert werden:
\begin{enumerate}[resume]
	\item mysql -u root -p < initDatabase.sql
\end{enumerate}
Damit sollte die Installation abgeschlossen sein und der Server kann über das Ausführen des Befehls „node server.js“ gestartet werden.