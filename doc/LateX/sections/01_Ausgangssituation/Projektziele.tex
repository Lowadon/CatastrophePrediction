\begin{enumerate}
	\item Mobiles ESP8622 \newline
	Als Station für die Aufzeichnung der Daten soll ein Microcontroller verwendet werden. Vorgabe der Stadt Tettnang war ein ESP 8266, welches über eine mobile Stromversorgung betrieben werden soll. 
	Als Sensoren für das Aufzeichnen der Daten sollen ein DHT22 und ein BMP 180 verwendet werden.
	
	\item DHT22 \newline
	Der DHT22 Sensor wird für die Aufzeichnung von Luftfeuchtigkeit und Temperatur verwendet.
	
	\item BMP180 \newline
	Für die Aufzeichnung der Höhe über Normal Null und Luftdruck wird ein BMP180 Sensor verwendet.
	
	\item Raspberry Pi 4 \newline
	Der Webserver braucht Hardware um betrieben zu werden. Da das Verkehrsvolumen auf dem Server als relativ gering eingeschätzt wird, sollte ein Raspberry Pi 4 als Hardwareinfrastruktur ausreichen.
	
	\item Webserver \newline
	Für das Bereitstellen von Visualisierungen der Daten und Aufzeichnung der Daten in einem Datenbank System wird ein Webserver als Interface benötigt. Dafür wird in diesem Projekt NodeJS verwendet.
	
	\item Frontend \newline
	Für die Darstellung der Daten soll eine Website zur Verfügung bereitgestellt werden. Diese soll vorerst eine einzelne Seite sein, die nur eine Tabelle die die Höhe über Normal Null auf einer Horizontalen Zeitachse darstellt.
	
	\item Backend \newline
	Das Backend soll das Topic im MQTT-Broker abonnieren, und dann alle Updates in die Datenbank speichern. Außerdem soll das Backend die Daten aus der Datenbank auslesen und für die Frontendimplementierung bereitstellen.
	
	\item Datenbankserver \newline
	Der Datenbankserver soll auch auf dem Raspberry Pi laufen und die Daten, die von den Sensoren ausgeliefert wurden, permanent abspeichern. Als Datenbanksystem soll MariaDB, eine SQL -Implementierung, verwendet werden.
	
	\item Kommunikation mit MQTT-Broker \newline
	Die Daten müssen zwischen den verschiedenen Geräten ausgetauscht werden. Für die Kommunikation wird im Vorfeld ein öffentlicher MQTT-Broker verwendet. In diesem Projekt wird voraussichtlich mit HiveMQ gearbeitet. Optimal wäre das Hosten eines privaten MQTT-Brokers, aber dies liegt außerhalb des Projektrahmens.
	
	\item Webserver und Broker \newline
	Der Webserver soll sich als Abonnent am entsprechenden Topic des MQTT-Brokers anmelden, um die regelmäßigen Updates des Topics und somit neue Daten zu erhalten.
\end{enumerate}