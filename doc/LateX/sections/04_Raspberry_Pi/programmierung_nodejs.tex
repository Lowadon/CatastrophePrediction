\subsubsection{Konfiguration der MariaDB Verbindung}
Das Code Snippet \ref{lst:conf_mariadb_conn} bescheibt die Datenbankverbindung zum lokal installierten Datenbankserver. Es enthält die benötigten Bibliotheken, den Userzugang und die Datenbank auf dem Server mit dem sich der Webserver verbindet.


\subsubsection{Konfiguration der MQTT Verbindung}
Der Code in \ref{lst:conf_mqtt_conn} bescheibt die Konfiguration des MQTT Objekts. Hier wurde nicht, wie geplant, mit HiveMQ gearbeitet, sondern mit Mosquitto. Getestet wurde auch der Eclipseprojects MQTT Broker. Für unser Projekt war aber Mosquitto der stabilste öffentliche Broker. Auch hier werden erst die benötigten Bibliotheken angegeben. Es folgen die Broker-URL, der Port und das Topic. Dann wird eine Verbindung aufgebaut und sich als Abonnent am Topic angemeldet, so dass der Broker Updates an den Webserver schicken kann.

\subsubsection{Event: MQTT Message}
Dieses Event \ref{lst:event_message} beschreibt was passiert, wenn eine Nachricht vom MQTT-Broker empfangen wird. Erst wird die Nachricht in die Konsole geschrieben, was beim Debugging hilft und dann wird die Nachricht in der Funktion: messageRecieved weiterverarbeitet.

\subsubsection{Funktion: messageRecieved}
In der Funktion messageRecieved(message) \ref{lst:function_message_recieved} wird, wie oben erwähnt, die Nachricht verarbeitet. Da die Daten in zwei verschiedene Tabellen in der Datenbank geschrieben werden müssen werden hier auch zwei SQL-Queries ausgeführt.

\subsubsection{Route: getData}
Diese Route \ref{lst:route_get_data} wird aufgerufen, wenn die Route: Index geladen wird. Dabei wird die asynchrone Funktion readData(req, res) ausgeführt.

\subsubsection{Funktion: readData}
Die Funktion readData(req, res) \ref{lst:function_read_data} liest Daten aus der Datenbank und sendet diese and die Website, damit sie dort visualisiert werden können.