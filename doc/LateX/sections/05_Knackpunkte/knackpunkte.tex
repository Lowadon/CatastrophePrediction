Die Knackpunkte im Projekt sind vorwiegend durch kleinere, jedoch schwer zu lokalisierende Fehler entstanden. \newline

Bei der Datenbank gab es initiale Schwierigkeiten mit dem Initialisierungsskript, welches das Setup und vorwiegend mehrfache Setups stark vereinfacht und beschleunigt.
Letztendlich ließen sich hier die Schwierigkeiten auf inkorrekte Syntax zurückführen, welche jedoch teilweise sehr schwer zu lokalisieren war. \newline
Auch bei der Schnittstelle zwischen der Datenbank und dem NodeJS-Backend gab es aufgrund von Syntaxfehlern mittlere Verzögerungen.
Letztendlich waren wir jedoch in der Lage, alle Fehler zu beheben. \newline
Beim ESP gab es ebenfalls kleinere Startschwierigkeiten, die Verbindung zum Broker konnte anfangs nicht hergestellt werden, was dann jedoch auf einen inkorrekten Port und eine fehlerhafte Klasseninstanziierung zurückzuführen war.
Da der ESP keine interne Zeit hat und seine Zeit immer von einem NTP-Server bezieht gab es auch hier kleinere Umstände.
Da der Hotspot eines Smartphones keinen NTP-Server direkt bereitstellt, musste das Auslesen der Zeit umprogrammiert werden, sodass nicht nur leere Timestamps zurückgeliefert wurden.