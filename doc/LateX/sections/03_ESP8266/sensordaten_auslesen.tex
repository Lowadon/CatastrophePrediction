Weil MicroPython auf dem ESP in Kombination mit einem MQTT-Broker schon 
zuvor im Unterricht Probleme bereitet hat, haben wir uns dazu entschieden, den ESP mit C++ zu programmieren.

Für das Handling der Sensordaten und des Timestamps, welcher später übermittelt werden soll, wurde eine separate Klasse "`sensorData"' angelegt.
Über den Konstruktor können die Werte an eine Klasseninstanz übergeben werden. 
Werden diese dann zum Übermitteln an den Broker benötigt, können die Werte dann einfach per Pfeiloperator ("`->"') abgefragt werden.

Für sowohl den DHT als auch den BMP wurde die "`Adafruit\_Sensor"'-Bibliothek verwendet.
Diese wird mit \textit{\#include Adafruit\_Sensor.h} eingebunden.

\subsubsection{Auslesen DHT}
	Beim DHT war es noch zusätzlich notwendig, die "`DHT"'-Bibliothek mit \textit{\#include <DHT.h>} einzubinden.
	Wenn die Bibliothek eingebunden ist, kann eine Instanz vom Typ "`DHT"' und der Bezeichnung "`dht"' mit der Zeile \textit{DHT dht(02,DHT22);} erzeugt werden.
	Die "`02"' beschreibt den Pin, auf welchen der DHT die Daten sendet und das "`DHT22"' beschreibt den genauen Typ des DHT.
	
	In der Funktion \textit{void setup()} wird mit der Zeile \textit{dht.begin();} das Auslesen des Pins gestartet. 
	Jetzt kann in der Funktion \textit{void loop()} mit \textit{dht.readTemperature()} die aktuelle Temperatur und mit \textit{dht.readHumidity()}
	die aktuelle Luftfeuchtigkeit ausgelesen werden.
	
\subsubsection{Auslesen BMP}
	Für den BMP musste noch die "`Adafruit\_BMP085"'-Bibliothek eingebunden werden.
	Hier wurde dann ebenfalls eine Klasseninstanz mit \textit{Adafruit\_BMP085 bmp180;} erzeugt. Beim BMP war keine weitere Konfiguration notwendig.
	
	Ähnlich zum DHT wird auch beim BMP das Auslesen mit \textit{bmp180.begin();} in der \textit{void setup()} gestartet.
	Da der Luftdruck jedoch je nach Wetterlage variieren kann, muss der Sensor am Start referenziert werden.
	Da die Höhe des Klassenzimmers bekannt ist (469 m über N.N.), wird immer beim Start mit \textit{seaLevelPressure = bmp180.readSealevelPressure(currentAltitude);} der aktuelle Luftdruck gemessen und auf den Luftdruck auf Meereshöhe rückgerechnet. Dieser wird dann in der Variable "`seaLevelPressure"' gespeichert, welche dann als Referenz für die weiteren Messungen verwendet wird.
	
	Da der BMP mit \textit{bmp180.readTemperature()} ebenfalls die aktuelle Temperatur messen kann, wird aus den Temperaturen des DHT und BMP der Mittelwert gebildet und abgespeichert.
	Die aktuelle Höhe kann über \textit{bmp180.readAltitude(seaLevelPressure)} in Anhängigkeit des zuvor referenzierten Luftdrucks gemessen werden.
	Da auch der Luftdruck auf der Website erscheinen soll, wird dieser mit \textit{(float)bmp180.readPressure()/100} ausgelesen.
	Die Konvertierung zu einem Float und das Dividieren durch 100 sorgt dafür, dass der Druck in mBar mit Nachkommastellen zurückgeliefert wird.

\subsubsection{Erzeugung Timestamp}
	Da der Timestamp keine zeitliche Abweichung zu den Messergebnissen haben soll, wird dieser annähernd zeitgleich ebenfalls durch den ESP abgespeichert.
	Aufgrund der Tatsache, dass der ESP über keine interne Uhr verfügt und auch \textit{std::chrono::system\_clock::now} nur auf den NTP-Server des verbundenen WLAN-Routers zugreift, kann diese Bibliothek nicht verwendet werden. Wenn der ESP mit dem Hotspot eines Smartphones verbunden wird, liefert \textit{std::chrono::system\_clock::now} nur noch "`0000000000"' zurück, da das Smartphone keinen NTP-Server zur Verfügung stellt.
	Diese Limitierung kann jedoch umgangen werden, indem direkt auf eine URL eines NTP-Servers verwiesen wird, von welchem dann die aktuelle Zeit bezogen werden kann.
	Für den direkten Verweis auf einen NTP-Server sind sowohl die "`NTPClient"' als auch die "`WiFiUdp"'-Bibliothek nötig.
	Die "`WiFiUdp"'-Instanz kann ohne weitere Parameter initialisiert werden (\textit{WiFiUDP ntpUDP;}), beim NTPClient müssen die WiFiUdp-Instanz, die Url des NTP-Servers und die Verschiebung zur UTC in Sekunden angegeben werden: \textit{NTPClient timeClient(ntpUDP, "pool.ntp.org", 7200);}.
	Auch der "`timeClient"' muss mit \textit{timeClient.begin();} gestartet werden, jedoch ist hier naheliegenderweise eine WLAN-Verbindung notwendig.
	Bevor die Zeit abgefragt wird, muss der Client mit \textit{timeClient.update();} geupdated werden, dass tatsächlich die aktuelle Zeit zurückgeliefert wird.
	Für einen Timestamp mit Uhrzeit und Datum muss zuerst die "`EpochTime"' abgefragt werden (\textit{timeClient.getEpochTime()}), dies liefert einen Unix-Timestamp zurück.
	Dieser kann dann mit \textit{strftime(timestamp, 20, "`\%Y-\%m-\%d \%H:\%M:\%S"' ,localtime(\&now));} in ein YYYY-MM-DD HH:MM:SS Format umformatiert und in die Variable "`timestamp"' abgespeichert werden.
	