% alle Abkürzungen, die in der Arbeit verwendet werden. Die Alphabetische Sortierung übernimmt Latex. Nachfolgend sind Beispiele genannt, welche nach Bedarf angepasst, gelöscht oder ergänzt werden können.

% Bei den unten stehenden Formelzeichen ist erläutert, wie explizite Sortierschlüssel über den Inhalt der eckigen Klammer angegeben werden.

% Allgemeine Abkürzungen %%%%%%%%%%%%%%%%%%%%%%%%%%%%
%\nomenclature{Abb.}{Abbildung}
%\nomenclature{bzw.}{beziehungsweise}
%\nomenclature{DHBW}{Duale Hochschule Baden-Württemberg}
%\nomenclature{ebd.}{ebenda}
%\nomenclature{et al.}{at alii}
%\nomenclature{etc.}{et cetera}
%\nomenclature{evtl.}{eventuell}
%\nomenclature{f.}{folgende Seite}
%\nomenclature{ff.}{fortfolgende Seiten}
%\nomenclature{ggf.}{gegebenenfalls}
%\nomenclature{Hrsg.}{Herausgeber}
%\nomenclature{Tab.}{Tabelle}
%\nomenclature{u. a.}{unter anderem}
%\nomenclature{usw.}{und so weiter}
%\nomenclature{vgl.}{vergleiche}
%\nomenclature{z. B.}{zum Beispiel}
%\nomenclature{z. T.}{zum Teil}
\nomenclature{ESP}{Espressif ESP8266}
\nomenclature{MQTT}{Message Queuing Telemetry Transport}
\nomenclature{DHT}{Digital Temperature and Humidity Sensor}
\nomenclature{BMP}{Barometric Pressure Sensor}
\nomenclature{N.N.}{Normal Null}
\nomenclature{NTP}{Network Time Protocol}
\nomenclature{UTC}{Coordinated Universal Time}



% Dateiendungen %%%%%%%%%%%%%%%%%%%%%%%%%%%%%%%%%%%%
%\nomenclature{EMF}{Enhanced Metafile}
%\nomenclature{JPG}{Joint Photographic Experts Group}
%\nomenclature{PDF}{Portable Document Format}
%\nomenclature{PNG}{Portable Network Graphics}
%\nomenclature{XML}{Extensible Markup Language}
\nomenclature{HTML}{Hyper Text Markup Language}
\nomenclature{CSS}{Cascading Style Sheet}
\nomenclature{EJS}{Embedded Javascript Templating}

% Formelzeichen %%%%%%%%%%%%%%%%%%%%%%%%%%%%%%%%%%%%



